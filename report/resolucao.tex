\section*{Trabalho/Resolução:}

\subsection*{Introdução}

\paragraph{}O e-fólio A da cadeira de Programação, pretendia a avaliação dos
conhecimentos adquiridos, através da implementação de um jogo original,
intitulado de "Lamberta". Para tal, foram fornecidos 4 exercícios com
objectivos diferentes, de modo a implementação ser iterativa.

\paragraph{}O presente relatório é dividido sobre notas de implementação por
alínea. Foram concluídas todas as alíneas com sucesso, com cada ficheiro de
código fonte foi também apropriadamente comentado.

\subsection{}

\paragraph{} Na alínea (a) foi requerido um programa que recebe-se de
input uma sequência binária, e que traduzi-se a mesma para impressão como se
fosse um tabuleiro.

\paragraph{} Para tal, definiu-se o mapeamento dos estados do jogo da
Lamberta:

\begin{center}
\begin{tabular}{c|c}
	Input & Carácter/Estado \\
	\hline
	0 & O \\
	1 & X \\
\end{tabular}
\end{center}

\paragraph{} Para resolução do exercício, além de se implementar a função
"MostraLamberta", implementaram-se:

\begin{itemize}
	\item enum lamberta\_estados: Um enumerado com os estado possíveis, e a
		salvaguarda de um estado desconhecido.
	\item CaracterParaLambertaEstado: Uma função que traduz do carácter
		binário da entrada para a representação dos estados.
	\item LambertaEstadoParaCaracter: Uma função que traduz da representação
		interna dos estados para o carácter de estado(X ou O).
\end{itemize}

\paragraph{}Sendo que funcionalidade era relativamente simples, não foram
implementados mais testes.

\newpage

\subsection{}

\paragraph{} Na alínea (b) foi requerido um programa que recebe-se o tamanho
do tabuleiro e o número de chamadas a descartar, de uma função de geração de
números pseudo aleatórios fornecidos pela docência(randaux). Com base nesta
função dever-se-ia preencher a representação do tabuleiro.

\paragraph{} Esquema de mapeamento em função de randaux, baseado na paridade
do seu retorno:


\begin{center}
\begin{tabular}{c|c}
	paridade do retorno de randaux & Estado \\
	\hline
	par & O \\
	ímpar & X \\
\end{tabular}
\end{center}

\paragraph{} Para a resolução do exercício implementou-se a função:
PreencheLamberta.

\paragraph{} Removeu-se a função CaracterParaLambertaEstado, uma vez que se
tornou desnecessária.

\newpage

\subsection{}

\paragraph{} Na alínea (c) foi requerido a implementação do jogo "Lamberta"
para dois jogadores humanos, sendo que recebem o input por turno.

\paragraph{} Para a implementação deste programa mudou-se a representação do
tabuleiro, do vector simples de inteiros anterior, para um tipo abstracto de
dados(TAD) que contém todo o estado do tabuleiro. Das operações implementadas
para este tipo incluiu-se também uma função que testa a validade de uma
jogada.

\paragraph{} Dados da estrutura do tabuleiro:

\begin{itemize}
\item tabuleiro(int*): armazenamento dos estados do tabuleiro
\item tamanho(int): tamanho do tabuleiro
\item jogada(int): contador da jogada
\item segmento\_maximo(int): tamanho máximo do segmento de uma dada jogada
\end{itemize}

\paragraph{} Separou-se a validação de uma jogada da aplicação da mesma, para
ter um padrão mais conveniente para gerir o fluxo de jogo.

\paragraph{} Além dos testes fornecidos, testou-se os limites todos da
validação de uma jogada, que se entende por índice base, tamanho, e a
combinação de ambos face ao tamanho do tabuleiro. Além de obviamente de um
dado segmento possuir uma célula com o estado X.

\newpage

\subsection{}

\paragraph{} Na alínea (d) foi requerido a implementação do jogo "Lamberta"
para dois jogadores autónomos, com base num algoritmo fornecido.

\paragraph{} Em comparação com a alínea (c), procedeu-se aos seguintes passos:

\begin{itemize}
\item Substituição da função que gere a entrada da jogada, por uma função
	que implementa ordenadamente as regras descritas do algoritmo.
\item Implementação de funções que implementassem cada regra, assim como
	funções utilitárias, que permitissem a sua implementação efectivamente,
	eficientemente e expressivamente.
\end{itemize}

\paragraph{} O único teste que foi feito para estender a cobertura garantida
pelos exemplos, foi o input "5 1", que garante a validação da regra 2.

